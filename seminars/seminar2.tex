\section*{Seminarul 2}

\subsection*{Versiunile limbajului C++}

Limbajul C++ este definit de un standard internațional, pentru care există câteva versiuni distincte: C++98, C++03, C++11, C++14, C++17, C++20 (ultimele două cifre indică anul în care au fost adoptate).

La curs vi se vor preda exclusiv noțiuni existente în C++98 (și doar acestea vor apărea și la examenul scris), dar la colocviu și la proiectul de laborator \textbf{aveți voie} să folosiți elemente din versiunile mai recente ale limbajului.

La seminar/laborator, unde este cazul, voi încerca să menționez funcționalitățile noi sau alternativele moderne la anumite elemente de limbaj.

\subsection*{Un tur prin C++}

Atunci când învățăm un nou limbaj de programare, este bine să avem o vedere de ansamblu a tuturor capabilităților acelui limbaj.

În cadrul acestui seminar, am trecut repede prin \href{https://isocpp.org/images/uploads/2-Tour-Basics.pdf}{primul PDF} din cartea \emph{A Tour of C++} a lui Bjarne Stroutstrup (creatorul limbajului C++), oferită gratuit (într-o variantă draft) pe pagina \url{https://isocpp.org/tour}. Am introdus/menționat:
\begin{itemize}
    \item Subprogramul principal \texttt{int main()};
    \item Afișarea pe ecran folosind \texttt{cout};
    \item Definirea de noi subprograme;
    \item Tipurile de date fundamentale (discutate deja în primul seminar) și operațiile pe care le putem efectua cu ele;
    \item Sintaxa de inițializare uniformă (introdusă în C++11);
    \item Definierea unei variabile cu tip de date dedus automat, folosind cuvântul cheie \texttt{auto} (introdus în C++11);
    \item Modificatorii \texttt{const} și \texttt{constexpr} (al doilea a fost introdus în C++11); 
    \item Structuri condiționate și repetitive de bază;
    \item Strucutră condiționată de tip \texttt{switch};
    \item Tablouri unidimensionale și pointeri;
    \item Referințe;
    \item Definirea de noi tipuri de date folosind \texttt{struct};
    \item Accesarea datelor membru/metodelor folosind „\texttt{.}”, respectiv „\texttt{->}”;
    \item Definirea de noi tipuri de date folosind \texttt{class};
    \item Modificatorii de acces \texttt{public} și \texttt{private};
    \item Constructori și liste de inițializare;
    \item Metode;
    \item Tipuri de date de tip enumerație (\texttt{enum}); sintaxa modernă de definire a enumerațiilor (\texttt{enum class}, introdus în C++);
    \item Declarări vs.\ definiții; definirea unor funcții/metode în fișiere separate;
    \item Namespace-uri;
\end{itemize}

\subsection*{Modelarea unei probleme folosind C++}

O resursă utilă în pregătirea pentru colocviu/examen sunt \href{https://github.com/DimaOanaTeodora/Tutoriat-POO-2022}{materialele de la tutoriatul de POO din 2022}. Printre altele, găsiți acolo și un link la o \href{https://github.com/dariabroscoteanu/OOP/tree/main/Lab\%20Exams}{listă de colocvii din anii trecuți}, cu enunțuri și rezolvări.

Am început să ne uităm peste și să rezolvăm subiectul de la \href{https://github.com/dariabroscoteanu/OOP/blob/main/Lab\%20Exams/Colocviu2017\%20-\%2015\%2621/POO\%20colocviu\%20vara\%202017.jpg}{colocviul de POO din vara lui 2017}. Tema a fost o aplicație de gestiune a unor proprietăți și a unor contracte pentru o agenție imobiliară.

Primul pas în rezolvarea unui colocviu este să identificăm \textbf{entitățile} pe care va trebui să le gestionăm (\emph{clasele}), respectiv \textbf{legăturile} dintre ele (cum le \emph{compunem}, dacă va fi nevoie de \emph{moștenire} etc.) și ce \textbf{funcționalități} au (\emph{metodele} pe care le vom defini în fiecare clasă).

Pe baza enunțului problemei, am identificat următoarele entități:
\begin{itemize}
    \item \textbf{Proprietate:} o proprietate gestionată de agenția imobiliară (o casă sau un apartament).

    Conține următoarele câmpuri:
    \begin{itemize}
        \item adresa (șir de caractere);
        \item suprafața locuibilă în metri pătrați (număr real);
        \item valoarea chiriei pe metru pătrat (număr real);
    \end{itemize}

    Am distins două tipuri de proprietăți: \emph{case} și \emph{apartamente}.
    \begin{itemize}
        \item Pentru case va trebui să reținem în plus numărul de niveluri (număr natural) și suprafața curții în metrii pătrați (număr real).
        \item Pentru apartamente va trebui să reținem în plus etajul la care se află (număr natural) și număr de camere (număr natural).
    \end{itemize}

    Din enunț reiese că pentru proprietăți va trebui să avem o metodă care să calculeze costul \textbf{chiriei lunare} (valoarea înainte de aplicarea discount-ului).

    \item \textbf{Contract de închiriere:} un contract încheiat între agenție și un chiriaș.

    Conține următoarele câmpuri:
    \begin{itemize}
        \item numele clientului (șir de caractere);
        \item perioada de închiriere (o „perioadă”, un tip de date care va fi definit de noi);
        \item o referință/un pointer la proprietatea care se închiriază;
    \end{itemize}

    Avem nevoie de o referință la proprietatea pe care vrem să o închiriem, deoarece la un moment dat va trebui să calculăm cât este \textbf{chiria lunară} a unei proprietăți, cu tot cu \textbf{discount}-ul aplicat, iar atunci vom vrea să ne folosim de metoda de calcul a chiriei lunare (fără discount) din clasa \texttt{Proprietate}.

    \item \textbf{Contract de vânzare-cumpărare:} un contract prin care agenția vinde o proprietate la un client, cu banii jos sau în rate.

    Conține următoarele câmpuri:
    \begin{itemize}
        \item numele clientului (șir de caractere);
        \item perioada de închiriere (o „perioadă”, un tip de date care va fi definit de noi);
        \item o referință/un pointer la proprietatea care se închiriază;
    \end{itemize}

    Și în acest caz reținem o referință la proprietatea care va fi vândută, din considerente similare; prețul de vânzare al proprietății depinde de chiria lunară, calculată de o metodă din clasa \texttt{Proprietate}.
    
    \item \textbf{Perioadă:} un interval calendaristic, cuprins între două date.
    
    \item \textbf{Dată:} reține luna și anul în care a avut loc un eveniment (în contextul cerinței, nu ni se cere să reținem și ziua).
\end{itemize}

Am observat de asemenea că există o oarecare similaritate între contractele de închiriere și cele de vânzare-cumpărare (ambele rețin numele unui client); am putea avea o singură entitate \textbf{Contract} (care să rețină numele clientului) și două sub-entități (cu date membru și metode specifice).
