\section*{Seminarul 4}

În cadrul acestui seminar vom discuta despre \emph{moștenire}, un concept din limbajul C++. Moștenirea este o metodă prin care putem \textbf{elimina duplicarea de cod}, reutilizând aceleași date membru și metode între mai multe tipuri de date, dar și prin care putem \textbf{introduce comportament nou}, controlând dinamic ce variantă a unei metode se execută cu ajutorul \emph{polimorfismului la execuție}.

\subsection*{Entitățile dintr-un model de date}

În timpul definirii modelului de date pentru o aplicație, pot apărea în mod natural câmpuri comune între obiecte. Moștenirea (cu modificatorul de acces \texttt{public}) ne permite să grupăm aceste date membru (și metodele asociate) într-o singură clasă, să-i dăm o denumire și să o refolosim.

În astfel de situații, între clasele moștenitoare și clasa de bază putem spune că avem o relație de tip „\emph{is a}” (ca să refolosim terminologia de la baze de date): \textit{\texttt{User} is an \texttt{Entity}}, \textit{\texttt{AdminUser} is a \texttt{User}}, \textit{\texttt{Product} is an \texttt{Entity}} etc.

\subsubsection*{Exemplu: Modelul de date pentru o aplicație de tip magazin}

Acest exemplu conține câteva fragmente de cod dintr-o aplicație care ar putea reține modelul de date pentru un magazin cu utilizatori și produse.

Clasa \texttt{Entity} reprezintă un tip de obiect din modelul nostru de date. Vrem ca toate entitățile să poată fi identificate printr-un ID numeric.
\begin{lstlisting}
class Entity
{ 
  int id;

public:
  int getId() const { return id; }
  void setId(int newId) { id = newId; }
};
\end{lstlisting}

Clasa \texttt{User} reține datele unui utilizator al aplicației.
\begin{lstlisting}
class User : public Entity
{
  string name, email;
  string role; int age;

public:
  const string& getEmail() const { return email; }
  void setEmail(const string& newEmail) 
  { email = newEmail; }

  // etc.
};
\end{lstlisting}

Clasa \texttt{AdminUser} extinde clasa \texttt{User} cu noi funcționalități. Este folosită doar pentru utilizatorii care au rol de admin.
\begin{lstlisting}
class AdminUser : public User
{
public:
  void doAdminStuff() { /* ... */ }
};
\end{lstlisting}

Clasa \texttt{Product} reține datele pentru un produs.
\begin{lstlisting}
class Product: public Entity
{
  string name, description;
  double price;

public:
  double getPrice() const { return price; }
  void setPrice(double newPrice) { price = newPrice; }

  // etc.
}
\end{lstlisting}

Nu am inclus în codul de mai sus constructorii sau metodele de citire/afișare, însă și acestea pot face referire la cele din clasele de bază, pentru a reduce la minim efortul de implementare. Când folosim moștenirea, este suficient să definim în plus doar comportamentul specific al fiecărei clase.

Un alt avantaj al acestui mod de lucru este că în multe limbaje de programare există unelte denumite \href{https://en.wikipedia.org/wiki/Object\%E2\%80\%93relational_mapping}{\textit{object-relational mappers}}, care pot face automat legătura între ierarhia voastră de clase și o bază de date. De exemplu, dacă veți ajunge să dezvoltați aplicații web în \href{https://dotnet.microsoft.com/en-us/apps/aspnet}{ASP.NET}, biblioteca \href{https://learn.microsoft.com/en-us/ef/}{Entity Framework} vă permite să generați automat tabele într-o bază de date SQL pe baza claselor voastre.

\subsection*{Implementări specifice platformei}

Pe lângă reutilizarea codului existent, moștenirea ne permite și să schimbăm în mod dinamic \emph{implementarea} pe care vrem să o folosim pentru un set de funcții, păstrând aceeași \emph{interfață} (apelăm aceleași metode, dar se execută implementări diferite ale lor).

\subsubsection*{Exemplu: Sistemul de grafică al unui game engine}

Acest exemplu arată cum s-ar putea defini sistemul de grafică într-un motor de jocuri, astfel încât să avem opțiunea de a schimba la execuție API-ul de grafică folosit.

Pentru a desena imagini pe calculator, trebuie să folosim un \href{https://en.wikipedia.org/wiki/API}{API} (\textit{Application Programming Interface}), prin care aplicația noastră comunică cu sistemul de operare și mai departe cu placa video a calculatorului. Dificultatea apare din multitudinea de API-uri care se folosesc în practică: pe Windows putem folosi \href{https://learn.microsoft.com/en-us/windows/win32/direct3d}{Direct3D}, pe MacOS și pe Linux putem folosi \href{https://www.opengl.org/}{OpenGL}, iar pe dispozitivele mobile se folosește de obicei \href{https://www.khronos.org/opengles/}{OpenGL ES} (Embedded Specification).

Clasa \texttt{GraphicsEngine} joacă rolul unei interfețe care descrie ce fel de metode trebuie oferite de fiecare implementare concretă a sistemului de grafică.
\begin{lstlisting}
class GraphicsEngine
{
protected:
  GraphicsEngine() { /* cod comun de initializare */ }

public:
  virtual ~GraphicsEngine()
  { /* cod comun de deinitializare */ }

  virtual void clearScreen() = 0;
  virtual void drawRectangle() = 0;

  // ...alte functii de desenare...
};
\end{lstlisting}

Avem trei implementări diferite pentru sistemul de grafică, fiecare utilizând un API diferit: Direct3D, OpenGL, respectiv OpenGL ES.
\begin{lstlisting}
class Direct3DEngine: public GraphicsEngine
{
public:
  void clearScreen()
  {
    // golesc fereastra folosind Direct3D
  }

  void drawRectangle()
  {
    // desenez un dreptunghi folosind Direct3D
  }
};

class OpenGLEngine: public GraphicsEngine
{
public:
  void clearScreen()
  {
    // golesc fereastra folosind OpenGL
  }

  void drawRectangle()
  {
    // desenez un dreptunghi folosind OpenGL
  }
};

class OpenGLESEngine: public GraphicsEngine
{
public:
  void clearScreen()
  {
    // golesc ecranul folosind OpenGL ES
  }

  void drawRectangle()
  {
    // desenez un dreptunghi folosind OpenGL ES
  }
};
\end{lstlisting}

După ce game engine-ul pornește și creează o fereastră (la fel, s-ar face prin API-uri specifice fiecărui sistem de operare), acesta inițializează sistemul de grafică potrivit situației și îl folosește pentru a desena pe ecran:
\begin{lstlisting}
int main()
{
  // ...codul de initializare pentru game engine...
  
  GraphicsEngine* graphicsEngine;
  if (isWindows)
  {
    // pe Windows folosesc D3D
    graphicsEngine = new Direct3DEngine();
  }
  else if (isMac || isLinux)
  {
    // pe Mac/Linux folosesc OpenGL
    graphicsEngine = new OpenGLEngine();
  }
  else if (isAndroid || isIOS)
  {
    // pe mobile folosesc OpenGL ES
    graphicsEngine = new OpenGLESEngine();
  }
  else // ...

  // dupa ce incepe jocul:
  graphicsEngine->clearScreen();
  graphicsEngine->drawRectangle();

  // ...

  // la final, eliberez resursele grafice folosite
  delete graphicsEngine;
}
\end{lstlisting}

Avantajul acestei separări între interfețe și implementări este că ne permite să utilizăm aceleași metode de afișare (\texttt{clearScreen} și \texttt{drawRectangle}), fără să ne intereseze (în principiu) ce API de grafică este folosit în spate. În momentul în care apelăm aceste metode pe un pointer de tip \texttt{GraphicsEngine}, care trimite către o implementare concretă a sistemului de grafică, beneficiem de \textbf{polimorfism la execuție}.

În acest exemplu, motorul de grafică este ales în mod determinist la pornirea programului, dar nimic nu ne previne să-l schimbăm pe baza unor setări citite dintr-un fișier. De exemplu, Unity \href{https://docs.unity3d.com/2019.4/Documentation/Manual/GraphicsAPIs.html}{permite schimbarea API-ului de grafică folosit pentru fiecare platformă din setările proiectului}.

\subsection*{Reutilizarea implementării}

Exemplele anterioare au folosit moștenirea cu modificatorul de acces \texttt{public}, pentru că acesta este cel mai comun scenariu de utilizare pentru moștenire (în limbaje precum C\# sau Java, este chiar singura posibilitate). În C++, se pot folosi și modificatorii \texttt{protected} sau \texttt{private}. În acest caz, clasa moștenitoare nu mai are o relație de tipul „\emph{is a}” cu clasa de bază, ci mai degrabă o relație de tipul \href{https://stackoverflow.com/a/1374362/5723188}{„\emph{is implemented in terms of}”}.

\subsubsection*{Exemplu: Extinderea unei clase din biblioteca standard}

Un scenariu în care vrem să moștenim o clasă, dar să ascundem metodele ei în exterior, este când vrem să o reutilizăm mai convenabil în implementarea unor noi funcționalități.

În codul de mai jos, ne-am făcut propria noastră clasă pentru o stivă alocată dinamic de numere întregi (în loc să folosim \href{https://www.geeksforgeeks.org/stack-in-cpp-stl/}{\texttt{stack<int>}} oferit de biblioteca standard). Valorile sunt stocate cu ajutorul clasei \texttt{vector<int>}.
\begin{lstlisting}
class IntStack : private vector<int>
{
public:
  using vector<int>::clear;

  void push(int x) { push_back(x); }

  int pop()
  {
    int top = back();
    pop_back();
    return top;
  }
};
\end{lstlisting}

Am fi putut la fel de bine să nu moștenim această clasă și în schimb să avem o dată membru de tip \texttt{vector<int>}, dar prin moștenire putem reutiliza implementarea funcției \texttt{clear} fără să o mai redefinim.

Pe instanțele clasei \texttt{IntStack} nu putem apela decât trei metode: \texttt{clear} (care apelează direct metoda \texttt{clear} din \texttt{vector<int>}), \texttt{push} și \texttt{pop}. Toate celelalte metode ale clasei de bază (care ne-ar permite să obținem lungimea vectorului, să inserăm elemente în mijlocul lui etc.) sunt inaccesibile în afara clasei \texttt{IntStack}.